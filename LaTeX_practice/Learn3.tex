\documentclass[11pt]{article}

\begin{document}
use of simple brackets:
$$(x+1)$$
$$3(x+2)$$
use of square brackets:
$$[x+1]$$
$$3[x+\sin{x+20}+\pi(\sqrt{x^{2.89}+\log_5{2x^3}})] $$

print curly brackets:
$$\{a,b,c\}$$

print dollar sign:
$$\$12.554$$

$$3(\frac{2}{5}) $$

increase the size of brackets according to its components:
$$3\left(\frac{2}{5}\right) $$
$$3\left[\frac{2}{5}\right] $$
$$3\left\{\frac{2}{5}\right\} $$
$$3\left|\frac{x}{5x^{2x^5}}\right| $$


$$(5x+2)\left(\frac{5x^2+9x^5+\pi}{\tan(x^6+x^5+x^4+x^3+x^2+x^1+1.22)}\right)$$
$$(5x+2)\left[\frac{5x^2+9x^5+\pi}{\tan\{x^6+x^5+x^4+x^3+x^2+x^1+1.22\}}\right]$$

$$3\left\{\frac{x}{5x^{2x^5}}\right\} $$

not required one side bracket then replace that bracket with dot, as such :
$$3\left\{\frac{x}{5x^{2x^5}}\right. $$

$$3\left.\frac{x}{5x^{2x^5}}\right|_{x=1} $$
\end{document}
